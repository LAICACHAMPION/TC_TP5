\documentclass[../../tc_tp5_main.tex]{subfiles}

\begin{document}

%capítulo
\chapter{Celda Sallen-Key}

En esta secci\'on, se implementar\'an dos filtros pasabajos haciendo uso de celdas Sallen-Key en cascada. Sobre los mismos, se analizar\'a su respuesta en frecuencia, impedancia de entrada e impedancia de salida, y la sensibilidad de los par\'ametros caracter\'isticos del filtro a desviaciones en los valores de los componentes que lo integran respecto de su valor nominal. \par

Para esto, haremos en primer lugar un an\'alisis te\'orico de las celdas Sallen-Key.


\section{Introducci\'on: la celda Sallen-Key}

La Sallen-Key es una celda que permite realizar un filtro de segundo orden utilizando s\'olo \textit{op amps}, resistencias y capacitores. Si bien normalmente con estos dos tipos de componentes pasivos s\'olo podr\'ian obtenerse polos reales, es decir $Q \leq \nicefrac{1}{2}$, el \textit{feedback} positivo introducido por el operacional permite obtener polos complejos conjugados, y por lo tanto una mayor selectividad. Como en este tipo de celdas existe un \'unico \textit{feedback} positivo, en general la sensibilidad del filtro a la dispersi\'on de los par\'ametros del operacional es menor, y a los valores de los componentes pasivos es mayor, respecto de otro tipo de celdas.\par  

\begin{figure}[H]
	\centering
	\begin{circuitikz}
  	\draw (0,0) node[op amp, yscale=-1] (opamp) {}
  		(opamp.-) -| (-1.5, -1.5) 
  		to [R = $R_3$, *-]  (-1.5, -3.5) node[ground] {}
  		
  		(-1.5, -1.5) to [R = $R_4$] (1.5, -1.5) 
  		to [short, -*] (1.5, 0) to [short, -o] (2, 0) node[right] {$V_{out}$}
  		(1.5,0) to [short] (opamp.out) 
  		
  		(opamp.+) to [short, -*] (-2.5, 0.5)
  		to [generic, l=$Z_3$] (-2.5, -1.5) node[ground]{}
  		
		(-2.5, 0.5) to [generic, l=$Z_2$] (-4.5, 0.5)
		to [generic, l=$Z_1$, *-o] (-6.5, 0.5) node[left]{$V_{in}$}  		
		
		(-4.5, 0.5) to [short] (-4.5, 2)
		to [generic, l=$Z_4$] (1.5,2)
		to [short] (1.5,0)
  	;
	\end{circuitikz}
	\caption{Celda Sallen-Key gen\'erica}
\end{figure}

En esta configuraci\'on, las resistencias $R_3$ y $R_4$ determinan la ganancia de la celda, puesto que forman un circuito no inversor en el camino del \textit{feedback} negativo, mientras que las dem\'as impedancias determinar\'an la ubicaci\'on de los polos y los ceros del circuito.\par


\begin{figure}[H]
	\centering
	\begin{circuitikz}
  	\draw (0,0) node[op amp, yscale=-1] (opamp) {}
  		(opamp.-) -| (-1.5, -1.5) 
  		to [R = $R_3$, *-]  (-1.5, -3.5) node[ground] {}
  		
  		(-1.5, -1.5) to [R = $R_4$] (1.5, -1.5) 
  		to [short, -*] (1.5, 0) to [short, -o] (2, 0) node[right] {$V_{out}$}
  		(1.5,0) to [short] (opamp.out) 	
  		
  		(opamp.+) to [short, -*] (-2.5, 0.5)
  		to [C, l_=$C_2$] (-2.5, -1.5) node[ground]{}
  		
		(-2.5, 0.5) to [R, l_=$R_2$] (-4.5, 0.5)
		to [R, l_=$R_1$, *-o] (-6.5, 0.5) node[left]{$V_{in}$}  		
		
		(-4.5, 0.5) to [short] (-4.5, 2)
		to [C, l=$C_1$] (1.5,2)
		to [short] (1.5,0)
  	;
	\end{circuitikz}
	\caption{Celda Sallen-Key pasabajos}
\end{figure}


\begin{figure}[H]
	\centering
	\begin{circuitikz}
  	\draw (0,0) node[op amp, yscale=-1] (opamp) {}
  		(opamp.-) -| (-1.5, -1.5) 
		 to [short] (1.5, -1.5) 
  		to [short, -*] (1.5, 0) to [short, -o] (2, 0) node[right] {$V_{out}$}
  		(1.5,0) to [short] (opamp.out) 

  		(opamp.+) to [short, -*] (-2.5, 0.5)
  		to [C, l_=$C_2$] (-2.5, -1.5) node[ground]{}
  		
		(-2.5, 0.5) to [R, l_=$R_2$] (-4.5, 0.5)
		to [R, l_=$R_1$, *-o] (-6.5, 0.5) node[left]{$V_{in}$}  		
		
		(-4.5, 0.5) to [short] (-4.5, 2)
		to [C, l=$C_1$] (1.5,2)
		to [short] (1.5,0)
  	;
	\end{circuitikz}
	\caption{Celda Sallen-Key pasabajos con ganancia unitaria}
\end{figure}


http://www.ti.com/lit/an/sloa024b/sloa024b.pdf

\end{document}
